%%%%%%%%%%%%%%%%%%%%%%%%%%%%%%%%%%%%%%%%%%%%%%%%
% ACT Cheat Sheet
% baposter Landscape Poster
% LaTeX Template
% Version 1.0 (11/06/13)
% baposter Class Created by:
% Brian Amberg (baposter@brian-amberg.de)
% This template has been downloaded from:
% http://www.LaTeXTemplates.com
% License:
% CC BY-NC-SA 3.0 (http://creativecommons.org/licenses/by-nc-sa/3.0/)
% Edited by Zachary Sherman
%%%%%%%%%%%%%%%%%%%%%%%%%%%%%%%%%%%%%%%%%%%%%%%%

%%%%%%%%%%%%%%%%%%%%%%%%%%%%%%%%%%%%%%%%%%%%%%%%
% NOTE
% The class file needed with this .tex file can be found here:
% http://www.brian-amberg.de/uni/poster/
%%%%%%%%%%%%%%%%%%%%%%%%%%%%%%%%%%%%%%%%%%%%%%%%

%----------------------------------------------------------------
%   PACKAGES AND OTHER DOCUMENT CONFIGURATIONS
%----------------------------------------------------------------

\documentclass[potrait, z1paper, fontscale=0.33]{baposter} % Adjust the font scale/size here
\title{ACT Cheat Sheet New}

\usepackage[utf8]{inputenc}

\usepackage{graphicx} % Required for including images
\graphicspath{{figures/}} % Directory in which figures are stored

\usepackage{xcolor}
\usepackage{colortbl}
\usepackage{tabu}

\usepackage{mathtools}
%\usepackage{amsmath} % For typesetting math
\usepackage{amssymb} % Adds new symbols to be used in math mode

\usepackage{booktabs} % Top and bottom rules for tables
\usepackage{enumitem} % Used to reduce itemize/enumerate spacing
\usepackage{palatino} % Use the Palatino font
\usepackage[font=small,labelfont=bf]{caption} % Required for specifying captions to tables and figures

\usepackage{multicol} % Required for multiple columns
\setlength{\columnsep}{1.5em} % Slightly increase the space between columns
\setlength{\columnseprule}{0mm} % No horizontal rule between columns

\usepackage{tikz} % Required for flow chart
\usetikzlibrary{decorations.pathmorphing}
\usetikzlibrary{shapes,arrows} % Tikz libraries required for the flow chart in the template

\newcommand{\compresslist}{ % Define a command to reduce spacing within itemize/enumerate environments, this is used right after \begin{itemize} or \begin{enumerate}
\setlength{\itemsep}{1pt}
\setlength{\parskip}{0pt}
\setlength{\parsep}{0pt}
}

\definecolor{lightblue}{rgb}{0.1,0.6, 3} % Defines the color used for content box headers

\begin{document}

\begin{poster}
{
headerborder=closed, % Adds a border around the header of content boxes
colspacing=0.8em, % Column spacing
bgColorOne=white, % Background color for the gradient on the left side of the poster
bgColorTwo=white, % Background color for the gradient on the right side of the poster
borderColor=lightblue, % Border color
headerColorOne=black, % Background color for the header in the content boxes (left side)
headerColorTwo=lightblue, % Background color for the header in the content boxes (right side)
headerFontColor=white, % Text color for the header text in the content boxes
boxColorOne=white, % Background color of the content boxes
textborder=roundedleft, % Format of the border around content boxes, can be: none, bars, coils, triangles, rectangle, rounded, roundedsmall, roundedright or faded
eyecatcher=true, % Set to false for ignoring the left logo in the title and move the title left
headerheight=0.06\textheight, % Height of the header
headershape=roundedright, % Specify the rounded corner in the content box headers, can be: rectangle, small-rounded, roundedright, roundedleft or rounded
headerfont=\Large\bf\textsc, % Large, bold and sans serif font in the headers of content boxes
textfont={\setlength{\parindent}{1.5em}}, % Uncomment for paragraph indentation
linewidth=2pt % Width of the border lines around content boxes
}
%----------------------------------------------------------------
%   Title
%----------------------------------------------------------------
{\bf\textsc{ACT Cheat Sheet}\vspace{0.5em}} % Poster title
{\textsc{\ A C T \ \ \ \ \ C h e a t \ \ \ \ \ S h e e t\ \hspace{12pt}}}
{\textsc{Learn More About ACT at https://anl-digr.github.io/ACT/ \hspace{12pt}}} 


%------------------------------------------------
% ACT Introduction
%------------------------------------------------
\headerbox{ACT Introduction}{name=introduction,column=0,row=0,span=1}{
\begin{flushleft}
The Atmospheric Commutity Toolkit (ACT) is a package for connecting Atmospheric data users to
the data. Has the ability to download, read, and visualize multi-file datasets from multiple
data sources. Currently, multi-panel timeseries plots are supported.
\end{flushleft}
}

%------------------------------------------------
% Installation
%------------------------------------------------
\headerbox{Installation}{name=installation,column=0,row=.124,span=1}{

\begin{flushleft}
\begin{tabular}{@{}ll@{}}
Installing ACT from source is the only way to get the\\
latest updates and enhancement to the software that\\
have no yet made it into a release. The latest source\\
code for ACT can be obtained from the GitHub\\
repository, https://github.com/ANL-DIGR/ACT.\\
Either download and unpack the zip file of the\\
source code or use git to checkout the repository:\\
\\
\-\hspace{0.1cm} \$ git clone https://github.com/ANL-DIGR/ACT.git\\
\\
\-\hspace{0.1cm} $\bullet$ To install in your home directory, use:\\
\-\hspace{0.1cm} \$ python setup.py install --user\\
\\
\-\hspace{0.1cm} $\bullet$ To install for all users on Unix/Linux:\\
\-\hspace{0.1cm} \$ python setup.py build\\
\-\hspace{0.1cm} \$ sudo python setup.py install\\
\\

\end{tabular}
\end{flushleft}
}

%------------------------------------------------
% Contact Information
%------------------------------------------------
\headerbox{Contact Information}{name=contact information,column=0,row=.395,span=1}{

\begin{flushleft}
\textbf{ACT GitHub Issues Forum:}
https://github.com/ANL-DIGR/ACT/issues
\end{flushleft}

\begin{flushleft}
\textbf{Email:}
\\
\-\hspace{0.4cm} atheisen@anl.gov\\
\-\hspace{0.4cm} rjackson@anl.gov\\
\end{flushleft}
}

%------------------------------------------------
% Contributing
%------------------------------------------------
\headerbox{Contributing}{name=contributing,column=0,row=.52,span=1}{

\begin{flushleft}
\begin{tabular}{@{}ll@{}}
ACT is an open source community software project.\\
Contributions to the package are welcomed from all\\
users.\\
\\
If you are planning on making changes that you\\
would like included in ACT, forking the repository is\\
highly recommended.\\
\\
We welcome contributions for all uses of ACT,\\
provided the code can be distributed under the\\
BSD 3-clause license. A copy of this license is\\
available in the LICENSE.txt file found at:\\
\\
\-\hspace{0.05cm} https://github.com/ANL-DIGR/ACT/blob/\\
\-\hspace{0.05cm} master/LICENSE.txt\\

\end{tabular}
\end{flushleft}
}


%------------------------------------------------
% Getting Started
%------------------------------------------------

\headerbox{Getting Started}{name=getting started,column=0,row=.765}{

\begin{flushleft}
\begin{tabular}{@{}ll@{}}
$>$$>$$>$ import act & To import ACT.\\
$>$$>$$>$ print(act.\_\_version\_\_) & Check version.
\end{tabular}
\end{flushleft}
}

%------------------------------------------------
% Corrections
%------------------------------------------------

\headerbox{Corrections}{name=corrections,column=0,row=.845}{

\begin{flushleft}
\begin{tabular}{@{}ll@{}}
$>$$>$$>$ obj = act.corrections.ceil.correct\_ciel(obj)\\
\-\hspace{0.2cm} $\bullet$ This procedure corrects celiometer data by filling\\
\-\hspace{0.5cm} all zero and negative values of backscatter with\\
\-\hspace{0.5cm} fill\_value and then converting the backscatter\\
\-\hspace{0.5cm} data into logarithmic space.\\
\\
\\
\end{tabular}
\end{flushleft}

}

%------------------------------------------------
% Discovery
%------------------------------------------------

\headerbox{Discovery}{name=discovery,column=1,row=0}{
\begin{flushleft}
\begin{tabular}{@{}ll@{}}
$>$$>$$>$ act.discovery.download\_data(\\
\-\hspace{1.2cm} username, token, datastream, startdate,\\
\-\hspace{1.2cm} enddate, output=None)\\
\-\hspace{0.2cm} $\bullet$ This programmatic interface allows users to query\\
\-\hspace{0.5cm} and automate machine-to-machine downloads of\\
\-\hspace{0.5cm} ARM data. This tool uses a REST URL and\\
\-\hspace{0.5cm} specific parameters (saveData, query), user ID\\
\-\hspace{0.5cm} and access token, a datastream name, a start date,\\
\-\hspace{0.5cm} and an end date, and data files matching the\\
\-\hspace{0.5cm} criteria will be returned to the user and\\
\-\hspace{0.5cm} downloaded.\\
\\
\-\hspace{0.2cm} $\bullet$ This will also eliminate the manual step of\\
\-\hspace{0.5cm} following a link in an email to download data.\\
\-\hspace{0.5cm} More information about the REST API and tools
\-\hspace{0.5cm} can be found on ARM Live:\\
\-\hspace{0.5cm} https://adc.arm.gov/armlive/\#scripts\\
\\
\-\hspace{0.2cm} $\bullet$ To login/register for an access token:\\
\-\hspace{0.5cm} https://adc.arm.gov/armlive/livedata/home\\
\end{tabular}
\end{flushleft}

}

%------------------------------------------------
% Input and Output Data
%------------------------------------------------

\headerbox{Input and Output Data}{name=input and output data,column=1,row=.295}{
\begin{flushleft}
\begin{tabular}{@{}ll@{}}

$>$$>$$>$ act\_obj = act.io.armfiles.read\_netcdf(\\
\-\hspace{1.2cm} filenames, concat\_dim='time',\\
\-\hspace{1.2cm} return\_None=False, **kwargs)\\
\-\hspace{0.2cm} $\bullet$ Returns xarray.Dataset with stored data and\\
\-\hspace{0.5cm} metadata from a user-defined query of ARM-\\
\-\hspace{0.5cm} standard netCDF files from a single datastream.\\
\\
$>$$>$$>$ flag = act.io.armfiles.check\_arm\_standards(\\
\-\hspace{1.2cm} act\_obj)\\
\-\hspace{0.2cm} $\bullet$ Checks to see if an xarray dataset conforms\\
\-\hspace{0.5cm} to ARM standards.\\
\\
$>$$>$$>$ act.io.dataset.ACTAccessor(act\_obj)\\
\-\hspace{0.2cm} $\bullet$ The xarray accessor for ACT data structures. This\\
\-\hspace{0.5cm} adds functionality that includes storing the times\\
\-\hspace{0.5cm} and names of each file in the dataset. In addition,\\
\-\hspace{0.5cm} the datastream can be given a name and a site.\\
\\
$>$$>$$>$ act.io.csvfiles.read\_csv(\\
\-\hspace{1.2cm} filename, sep=', ', engine='python',\\
\-\hspace{1.2cm} column\_names=None, skipfooter=0,\\
\-\hspace{1.2cm} **kwargs)\\
\-\hspace{0.2cm} $\bullet$ Returns an xarray.Dataset with stored data and\\
\-\hspace{0.5cm} metadata from user-defined query of CSV files.\\
\\
$>$$>$$>$ clean\_dataset = act.io.clean.CleanDataset(\\
\-\hspace{1.2cm} act\_obj)\\
\-\hspace{0.2cm} $\bullet$ Class containing functions for cleaning\\
\-\hspace{0.5cm} dataset. More on the functions below after\\
\-\hspace{0.5cm} defining the clean dataset object.\\
$>$$>$$>$ clean\_dataset.clean\_arm\_qc(\\
\-\hspace{1.2cm} override\_cf\_flag=True,\\
\-\hspace{1.2cm} clean\_units\_string=True,\\
\-\hspace{1.2cm} correct\_valid\_min\_max=True)\\
\-\hspace{0.2cm} $\bullet$ Function to clean up xarray object QC variables.\\
$>$$>$$>$ clean\_dataset.clean\_arm\_state\_variables(\\
\-\hspace{1.2cm} variables, override\_cf\_flag=True,\\
\-\hspace{1.2cm} clean\_units\_string=True, integer\_flag=True\\
\-\hspace{0.2cm} $\bullet$ Function to clean up state variables to use\\
\-\hspace{0.5cm} more CF style.\\
$>$$>$$>$ clean\_dataset.cleanup(\\
\-\hspace{1.2cm} cleanup\_arm\_qc=True,\\
\-\hspace{1.2cm} clean\_arm\_state\_vars=None,\\
\-\hspace{1.2cm} handle\_missing\_value=True,\\
\-\hspace{1.2cm} link\_qc\_variables=True, **kwargs\\
\-\hspace{0.2cm} $\bullet$ Wrapper method to automatically call all the\\
\-\hspace{0.5cm} standard methods for obj cleanup.\\
\\
\\
\\
\end{tabular}
\end{flushleft}

}

%------------------------------------------------
% Plotting
%------------------------------------------------

\headerbox{Plotting}{name=plotting,column=2,span=1,row=0}{
\begin{flushleft}

\begin{tabular}{@{}ll@{}}
\multicolumn{2}{l}{\cellcolor[HTML]{DDFFFF}\bf Display} \\
\\
Class that contains the common attributes and\\
routine between the differing Display classes.\\
\\
$>$$>$$>$ display = act.plotting.Display(\\
\-\hspace{1.2cm} obj, subplot\_shape=(1, ), ds\_name=None,\\
\-\hspace{1.2cm} subplot\_kw=None, **kwargs)\\
\\
$>$$>$$>$ display.add\_colorbar(\\
\-\hspace{1.2cm} mappable, title=None, subplot\_index=(0, ))\\
\-\hspace{0.2cm} $\bullet$ Adds a colorbar.\\
$>$$>$$>$ display.add\_subplots(\\
\-\hspace{1.2cm} subplot\_shape=(1, ), subplot\_kw=None,\\
\-\hspace{1.2cm} **kwargs)\\
\-\hspace{0.5cm} $\bullet$ Adds subplot to display object.\\
$>$$>$$>$ display.assign\_to\_figure\_axis(fig, ax)\\
\-\hspace{0.2cm} $\bullet$ This assigns the Display to a specific figure\\
\-\hspace{0.5cm} and axis.\\
$>$$>$$>$ display.put\_display\_in\_subplot(\\
\-\hspace{1.2cm} display, subplot\_index))\\
\-\hspace{0.2cm} $\bullet$ This will place a Display object into a specific\\
\-\hspace{0.5cm} subplot. 
\end{tabular}

\begin{tabular}{@{}ll@{}}
\\
\multicolumn{2}{l}{\cellcolor[HTML]{DDFFFF}\bf TimeSeriesDisplay} \\
\\
This subclass contains routines that are specific to\\
plotting time series plots from data.\\
\\
$>$$>$$>$ dis = act.plotting.TimeSeriesDisplay(\\
\-\hspace{1.2cm} obj, subplot\_shape=(1, ), ds\_name=None,\\
\-\hspace{1.2cm} **kwargs)\\
\\
$>$$>$$>$ dis.plot(field[, ...])\\
\-\hspace{0.2cm} $\bullet$ Makes a timeseries plot.\\
$>$$>$$>$ display.plot\_barbs\_from\_spd\_dir(dir\_field[, ...]\\
\-\hspace{0.2cm} $\bullet$ This procedure will make a wind barb plot\\
\-\hspace{0.5cm} timeseries.\\
$>$$>$$>$ dis.plot\_barbs\_from\_u\_v(u\_field, v\_field\\
\-\hspace{1.2cm}  [, ...])\\
\-\hspace{0.2cm} $\bullet$ This function will plot a wind barb timeseries\\
\-\hspace{0.5cm} from u and v wind data. If pres\_field is given, a\\
\-\hspace{0.5cm} height a time-height series will be plotted\\
\-\hspace{0.5cm} from 1-D wind data.\\
$>$$>$$>$ dis.plot\_time\_height\_xsection\_from\_1d\_data(\\
\-\hspace{1.2cm} data\_field, pres\_field[, ...])\\
\-\hspace{0.2cm} $\bullet$ This will plot a time-height cross section\\
\-\hspace{0.5cm} from 1D datasets using nearest neighbor\\
\-\hspace{0.5cm} interpolation on a regular time by height grid.\\
$>$$>$$>$ dis.time\_height\_scatter(\\
\-\hspace{1.2cm} data\_field=None[, ...])\\
\-\hspace{0.2cm} $\bullet$ Create a time series plot of altitude and data\\
\-\hspace{0.5cm} variable with color also indicating value with a\\
\-\hspace{0.5cm} color bar.\\
\end{tabular}

\begin{tabular}{@{}ll@{}}
\\
\multicolumn{2}{l}{\cellcolor[HTML]{DDFFFF}\bf SkewTDisplay} \\
\\
A class for making Skew-T plots.\\
\\
$>$$>$$>$ display = act.plotting.SkewTDisplay(\\
\-\hspace{1.2cm} obj, subplot\_shape=(1, ), ds\_name=None,\\
\-\hspace{1.2cm} **kwargs)\\
\\
$>$$>$$>$ display.add\_subplots(\\
\-\hspace{1.2cm} subplot\_shape=(1, ), **kwargs)\\
\-\hspace{0.2cm} $\bullet$ Adds subplots to the Display object. The\\
\-\hspace{0.5cm} current figure in the object will be deleted\\
\-\hspace{0.5cm} and overwritten.\\
$>$$>$$>$ display.plot\_from\_spd\_and\_dir(\\
\-\hspace{1.2cm} spd\_field, dir\_field, p\_field, t\_field,\\
\-\hspace{1.2cm} td\_field[, ...])\\
\-\hspace{0.2cm} $\bullet$ This plot will make a sounding plot from wind\\
\-\hspace{0.5cm} data that is given in speed and direction.\\

\end{tabular}

\end{flushleft}

}

\end{poster}
\newpage

%%%%%%%%%%%%%%%%%%%%%%%%%%%%%%%%%%%%%%%%%%%%%%%%%%%%%%%%%%
%%%%%%%%%%%%%%%%%%    SECOND PAGE    %%%%%%%%%%%%%%%%%%%%%
%%%%%%%%%%%%%%%%%%%%%%%%%%%%%%%%%%%%%%%%%%%%%%%%%%%%%%%%%%

\begin{poster}
{
headerborder=closed, colspacing=0.8em, bgColorOne=white, bgColorTwo=white, borderColor=lightblue, headerColorOne=black, headerColorTwo=lightblue, 
headerFontColor=white, boxColorOne=white, textborder=roundedleft, eyecatcher=true, headerheight=0.06\textheight, headershape=roundedright, headerfont=\Large\bf\textsc, linewidth=2pt 
}
%----------------------------------------------------------------
%   Title
%----------------------------------------------------------------
{\bf\textsc{ACT Cheat Sheet}\vspace{0.5em}} % Poster title
{\textsc{\ A C T \ \ \ \ \ C h e a t \ \ \ \ \ S h e e t\ \hspace{12pt}}}
{\textsc{Learn More About ACT at https://anl-digr.github.io/ACT/ \hspace{12pt}}}  

%------------------------------------------------
% Plotting Continued
%------------------------------------------------

\headerbox{Plotting}{name=plotting,column=0,span=1,row=0}{
\begin{flushleft}
\begin{tabular}{@{}ll@{}}
\multicolumn{2}{l}{\cellcolor[HTML]{DDFFFF}\bf SkewTDisplay Continued}\\
\\
$>$$>$$>$ display.plot\_from\_u\_and\_v(\\
\-\hspace{1.2cm} u\_field, v\_field, p\_field, t\_field,\\
\-\hspace{1.2cm} td\_field[, ...])\\
\-\hspace{0.2cm} $\bullet$ This function will plot a Skew-T from a\\
\-\hspace{0.5cm} sounding dataset. The wind data must be given\\
\-\hspace{0.5cm} in u and v.
\end{tabular}
\\
\begin{tabular}{@{}ll@{}}
\\
\multicolumn{2}{l}{\cellcolor[HTML]{DDFFFF}\bf WindRoseDisplay} \\
\\
A class for handing wind rose plots..\\
\\
$>$$>$$>$ display = act.plotting.WindRoseDisplay(\\
\-\hspace{1.2cm} obj, subplot\_shape=(1, ), ds\_name=None,\\
\-\hspace{1.2cm} **kwargs)\\
\\
$>$$>$$>$ display.plot(dir\_field, spd\_field[, ...])\\
\-\hspace{0.2cm} $\bullet$ Makes the wind rose plot from the given dataset.\\
\end{tabular}

\begin{tabular}{@{}ll@{}}
\\
\multicolumn{2}{l}{\cellcolor[HTML]{DDFFFF}\bf XSectionDisplay} \\
\\
Plots cross sections of multidimensional datasets.\\
\\
$>$$>$$>$ display = act.plotting.XSectionDisplay(\\
\-\hspace{1.2cm} obj, subplot\_shape=(1, ), ds\_name=None,\\
\-\hspace{1.2cm} **kwargs)\\
\\
$>$$>$$>$ display.plot\_xsection(dsname, varname[, ...])\\
\-\hspace{0.2cm} $\bullet$ This function plots a cross section whose x and\\
\-\hspace{0.5cm} y coordinates are specified by the variable names\\
\-\hspace{0.5cm} either provided by the user or automatically\\
\-\hspace{0.5cm} detected by xarray.\\
$>$$>$$>$ display.plot\_xsection\_map(\\
\-\hspace{1.2cm} dsname, varname[, ...])\\
\-\hspace{0.2cm} $\bullet$ Plots a cross section of 2D data on a geographical\\
\-\hspace{0.5cm} map.\\
\end{tabular}

\end{flushleft}

}

%------------------------------------------------
% Retrievals
%------------------------------------------------

\headerbox{Retrievals}{name=retrievals,column=0,span=1,row=.527}{
\begin{flushleft}
\begin{tabular}{@{}ll@{}}
$>$$>$$>$ ds = act.retrievals.calculate\_stability\_indicies(\\
\-\hspace{1.2cm} ds, temp\_name='temperature',\\
\-\hspace{1.2cm} td\_name='dewpoint\_temperature',\\
\-\hspace{1.2cm} p\_name='pressure', moving\_ave\_window=0)\\
\-\hspace{0.2cm} $\bullet$ Calculates stability indices and adds it\\
\-\hspace{0.5cm} to the data set.
\end{tabular}
\end{flushleft}

}

%------------------------------------------------
% Utilities
%------------------------------------------------

\headerbox{Utilities}{name=utilities,column=0,span=1,row=0.65}{
\begin{flushleft}
\begin{tabular}{@{}ll@{}}
$>$$>$$>$ dates = act.utils.datetime\_utils.dates\_between(\\
\-\hspace{1.2cm} sdate, edate)\\
\-\hspace{0.2cm} $\bullet$ Ths procedure returns all of the dates between\\
\-\hspace{0.5cm} sdate and edate.\\
\\
$>$$>$$>$ time, data = act.utils.data\_utils.add\_in\_nan(\\
\-\hspace{1.2cm} time, data)\\
\-\hspace{0.2cm} $\bullet$ This procedure adds in NaNs for given time\\
\-\hspace{0.5cm} periods in time when there is no corresponding\\
\-\hspace{0.5cm} data available. This is useful for timeseries that\\
\-\hspace{0.5cm} have irregular gaps in data.\\
\\
$>$$>$$>$ val = act.utils.data\_utils.get\_missing\_value(\\
\-\hspace{1.2cm} variable, default=-9999,\\
\-\hspace{1.2cm} add\_if\_missing\_in\_obj=False,\\
\-\hspace{1.2cm} use\_FillValue=False, nodefault=False)\\
\-\hspace{0.2cm} $\bullet$ Method to get missing value from missing\_value\\
\-\hspace{0.5cm} or \_FillValue attribute.\\
\\
$>$$>$$>$ ds = act.utils.data\_utils.assign\_coordinates(\\
\-\hspace{1.2cm} ds, coord\_list)\\
\-\hspace{0.2cm} $\bullet$ This procedure will create a new ACT dataset\\
\-\hspace{0.5cm} whose coordinates are designated to be the\\
\-\hspace{0.5cm} variables in a given list. 
\end{tabular}

\end{flushleft}

}

\end{poster}
\end{document}
