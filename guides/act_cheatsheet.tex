%%%%%%%%%%%%%%%%%%%%%%%%%%%%%%%%%%%%%%%%%%%%%%%%
% ACT Cheat Sheet
% baposter Landscape Poster
% LaTeX Template
% Version 1.0 (11/06/13)
% baposter Class Created by:
% Brian Amberg (baposter@brian-amberg.de)
% This template has been downloaded from:
% http://www.LaTeXTemplates.com
% License:
% CC BY-NC-SA 3.0 (http://creativecommons.org/licenses/by-nc-sa/3.0/)
% Edited by Zachary Sherman
%%%%%%%%%%%%%%%%%%%%%%%%%%%%%%%%%%%%%%%%%%%%%%%%

%%%%%%%%%%%%%%%%%%%%%%%%%%%%%%%%%%%%%%%%%%%%%%%%
% NOTE
% The class file needed with this .tex file can be found here:
% http://www.brian-amberg.de/uni/poster/
%%%%%%%%%%%%%%%%%%%%%%%%%%%%%%%%%%%%%%%%%%%%%%%%

%----------------------------------------------------------------
%   PACKAGES AND OTHER DOCUMENT CONFIGURATIONS
%----------------------------------------------------------------

\documentclass[potrait, z1paper, fontscale=0.33]{baposter} % Adjust the font scale/size here
\title{ACT Cheat Sheet New}

\usepackage[utf8]{inputenc}

\usepackage{graphicx} % Required for including images
\graphicspath{{figures/}} % Directory in which figures are stored

\usepackage{xcolor}
\usepackage{colortbl}
\usepackage{tabu}

\usepackage{mathtools}
%\usepackage{amsmath} % For typesetting math
\usepackage{amssymb} % Adds new symbols to be used in math mode

\usepackage{booktabs} % Top and bottom rules for tables
\usepackage{enumitem} % Used to reduce itemize/enumerate spacing
\usepackage{palatino} % Use the Palatino font
\usepackage[font=small,labelfont=bf]{caption} % Required for specifying captions to tables and figures

\usepackage{multicol} % Required for multiple columns
\setlength{\columnsep}{1.5em} % Slightly increase the space between columns
\setlength{\columnseprule}{0mm} % No horizontal rule between columns

\usepackage{tikz} % Required for flow chart
\usetikzlibrary{decorations.pathmorphing}
\usetikzlibrary{shapes,arrows} % Tikz libraries required for the flow chart in the template

\newcommand{\compresslist}{ % Define a command to reduce spacing within itemize/enumerate environments, this is used right after \begin{itemize} or \begin{enumerate}
\setlength{\itemsep}{1pt}
\setlength{\parskip}{0pt}
\setlength{\parsep}{0pt}
}

\definecolor{lightblue}{rgb}{0.1,0.6, 3} % Defines the color used for content box headers

\begin{document}

\begin{poster}
{
headerborder=closed, % Adds a border around the header of content boxes
colspacing=0.8em, % Column spacing
bgColorOne=white, % Background color for the gradient on the left side of the poster
bgColorTwo=white, % Background color for the gradient on the right side of the poster
borderColor=lightblue, % Border color
headerColorOne=black, % Background color for the header in the content boxes (left side)
headerColorTwo=lightblue, % Background color for the header in the content boxes (right side)
headerFontColor=white, % Text color for the header text in the content boxes
boxColorOne=white, % Background color of the content boxes
textborder=roundedleft, % Format of the border around content boxes, can be: none, bars, coils, triangles, rectangle, rounded, roundedsmall, roundedright or faded
eyecatcher=true, % Set to false for ignoring the left logo in the title and move the title left
headerheight=0.06\textheight, % Height of the header
headershape=roundedright, % Specify the rounded corner in the content box headers, can be: rectangle, small-rounded, roundedright, roundedleft or rounded
headerfont=\Large\bf\textsc, % Large, bold and sans serif font in the headers of content boxes
textfont={\setlength{\parindent}{1.5em}}, % Uncomment for paragraph indentation
linewidth=2pt % Width of the border lines around content boxes
}
%----------------------------------------------------------------
%   Title
%----------------------------------------------------------------
{\bf\textsc{ACT Cheat Sheet}\vspace{0.5em}} % Poster title
{\textsc{\ A C T \ \ \ \ \ C h e a t \ \ \ \ \ S h e e t\ \hspace{12pt}}}
{\textsc{Learn More About ACT at https://arm-doe.github.io/ACT/ \hspace{12pt}}}


%------------------------------------------------
% ACT Introduction
%------------------------------------------------
\headerbox{ACT Introduction}{name=introduction,column=0,row=0,span=1}{
\begin{flushleft}
The Atmospheric data Community Toolkit (ACT) is an open source Python
toolkit for working with atmospheric time-series datasets of varying
dimensions. The toolkit has functions for every part of the scientific
process; discovery, IO, quality control, corrections, retrievals,
visualization, and analysis. It is a community platform for sharing
code with the goal of reducing duplication of effort and better
connecting the science community with programs such as the Atmospheric
Radiation Measurement (ARM) User Facility.
\end{flushleft}
}

%------------------------------------------------
% Installation
%------------------------------------------------
\headerbox{Installation}{name=installation,column=0,row=.177,span=1}{

\begin{flushleft}
\begin{tabular}{@{}ll@{}}
Installing ACT from source is the only way to get the\\
latest updates and enhancement to the software that\\
have no yet made it into a release. The latest source\\
code for ACT can be obtained from the GitHub\\
repository, https://github.com/ARM-DOE/ACT.\\
Either download and unpack the zip file of the\\
source code or use git to checkout the repository:\\
\\
\-\hspace{0.1cm} \$ git clone https://github.com/ARM-DOE/ACT.git\\
\\
\-\hspace{0.1cm} $\bullet$ To install, use:\\
\-\hspace{0.1cm} \$ python setup.py install\\
\\
To install ACT using Anaconda or Miniconda, create\\
an environment and activate it:\\
\\
\-\hspace{0.4cm} $\bullet$ Then create a conda environment:\\
\-\hspace{0.4cm} \$ conda create -n act python=3.9\\
\\
\-\hspace{0.4cm} $\bullet$ Activate the ACT environment:\\
\-\hspace{0.4cm} \$ conda activate act\\
\\
\-\hspace{0.4cm} $\bullet$ Then install ACT:\\
\-\hspace{0.4cm} \$ conda install -c conda-forge act-atmos\\
\\

\end{tabular}
\end{flushleft}
}

%------------------------------------------------
% Contact Information
%------------------------------------------------
\headerbox{Contact Information}{name=contact information,column=0,row=.537,span=1}{

\begin{flushleft}
\textbf{ACT GitHub Issues Forum:}
https://github.com/ARM-DOE/ACT/issues
\end{flushleft}

\begin{flushleft}
\textbf{Email:}
\\
\-\hspace{0.4cm} atheisen@anl.gov\\
\-\hspace{0.4cm} mgrover@anl.gov\\
\-\hspace{0.4cm} zsherman@anl.gov\\
\-\hspace{0.4cm} rjackson@anl.gov\\
\end{flushleft}
}

%------------------------------------------------
% Contributing
%------------------------------------------------
\headerbox{Contributing}{name=contributing,column=0,row=.674,span=1}{

\begin{flushleft}
\begin{tabular}{@{}ll@{}}
ACT is an open source community software project.\\
Contributions to the package are welcomed from all\\
users.\\
\\
If you are planning on making changes that you\\
would like included in ACT, forking the repository is\\
highly recommended.\\
\\
We welcome contributions for all uses of ACT,\\
provided the code can be distributed under the\\
BSD 3-clause license. A copy of this license is\\
available in the LICENSE.txt file found at:\\
\\
\-\hspace{0.05cm} https://github.com/ARM-DOE/ACT/blob/\\
\-\hspace{0.05cm} master/LICENSE.txt\\

\end{tabular}
\end{flushleft}
}


%------------------------------------------------
% Getting Started
%------------------------------------------------

\headerbox{Getting Started}{name=getting started,column=0,row=.906}{

\begin{flushleft}
\begin{tabular}{@{}ll@{}}
$>$$>$$>$ import act & To import ACT.\\
$>$$>$$>$ print(act.\_\_version\_\_) & Check version.\\
\\
\end{tabular}
\end{flushleft}
}

%------------------------------------------------
% Corrections
%------------------------------------------------

\headerbox{Corrections}{name=corrections,column=1,row=0}{

\begin{flushleft}
\begin{tabular}{@{}ll@{}}
$>$$>$$>$ obj = act.corrections.correct\_ceil(obj)\\
\-\hspace{0.2cm} $\bullet$ This procedure corrects celiometer data.\\
\\
$>$$>$$>$ obj = act.corrections.correct\_dl(obj)\\
\-\hspace{0.2cm} $\bullet$ This procedure corrects doppler lidar data.\\
\\
$>$$>$$>$ obj = act.corrections.correct\_mpl(obj)\\
\-\hspace{0.2cm} $\bullet$ This procedure corrects MPL data.\\
\\
$>$$>$$>$ obj = act.corrections.correct\_rl(obj)\\
\-\hspace{0.2cm} $\bullet$ This procedure corrects raman lidar data.\\
\\
$>$$>$$>$ obj = act.corrections.correct\_wind(obj)\\
\-\hspace{0.2cm} $\bullet$ This procedure corrects wind speed and direction.\\
\-\hspace{0.5cm} for ship motion based on equations from NOAA\\
\-\hspace{0.5cm} tech.\\
\\
\end{tabular}
\end{flushleft}

}

%------------------------------------------------
% Discovery
%------------------------------------------------

\headerbox{Discovery}{name=discovery,column=1,row=.26}{
\begin{flushleft}
\begin{tabular}{@{}ll@{}}
$>$$>$$>$ act.discovery.download\_data(\\
\-\hspace{1.2cm} username, token, datastream, startdate,\\
\-\hspace{1.2cm} enddate[,...])\\
\-\hspace{0.2cm} $\bullet$ This programmatic interface allows users to query\\
\-\hspace{0.5cm} and automate machine-to-machine downloads of\\
\-\hspace{0.5cm} ARM data. This tool uses a REST URL and\\
\-\hspace{0.5cm} specific parameters (saveData, query), user ID\\
\-\hspace{0.5cm} and access token, a datastream name, a start date,\\
\-\hspace{0.5cm} and an end date, and data files matching the\\
\-\hspace{0.5cm} criteria will be returned to the user and\\
\-\hspace{0.5cm} downloaded.\\
\\
\-\hspace{0.2cm} $\bullet$ This will also eliminate the manual step of\\
\-\hspace{0.5cm} following a link in an email to download data.\\
\-\hspace{0.5cm} More information about the REST API and tools\\
\-\hspace{0.5cm} can be found on ARM Live:\\
\-\hspace{0.5cm} https://adc.arm.gov/armlive/\#scripts\\
\\
\-\hspace{0.2cm} $\bullet$ To login/register for an access token:\\
\-\hspace{0.5cm} https://adc.arm.gov/armlive/livedata/home\\
\\
$>$$>$$>$ act.discovery.croptype(lat, lon, year)\\
\-\hspace{0.2cm} $\bullet$ Function for working with the CropScape\\
\-\hspace{0.5cm} API to get a crop type based on the lat,lon, and\\
\-\hspace{0.5cm} year entered.\\
\\
$>$$>$$>$ act.discovery.download\_noaa\_psl\_data(\\
\-\hspace{1.5cm} site, instrument[, ...])\\
\-\hspace{0.2cm} $\bullet$ Function to download data from the NOAA\\
\-\hspace{0.5cm} PSL Profiler Network Data Library\\
\-\hspace{0.5cm} https://psl.noaa.gov/data/obs/datadisplay/\\
\\
$>$$>$$>$ act.discovery.get\_airnow\_forecast(token[,...])\\
\-\hspace{0.2cm} $\bullet$ This tool will get current or historical\\
\-\hspace{0.5cm} AQI values and categories for a reporting area\\
\-\hspace{0.5cm} by either Zip code or Lat/Lon coordinate.\\
\\
$>$$>$$>$ act.discovery.get\_airnow\_obs(token[,...])\\
\-\hspace{0.2cm} $\bullet$ This tool will get current or historical obs\\
\-\hspace{0.5cm} AQI values and categories for a reporting area\\
\-\hspace{0.5cm} by either Zip code or Lat/Lon coordinate.\\
\\
$>$$>$$>$ act.discovery.get\_asos(time\_window[,...])\\
\-\hspace{0.2cm} $\bullet$ Returns all of the station observations from\\
\-\hspace{0.5cm} the Iowa Mesonet from either a given latitude\\
\-\hspace{0.5cm} and longitude window or a given station code.\\
\\
$>$$>$$>$ act.discovery.get\_airnow\_bounded\_obs(\\
\-\hspace{1.5cm} token[,...])\\
\-\hspace{0.2cm} $\bullet$ Get AQI values or data concentrations\\
\-\hspace{0.5cm} for a specific date and time range and set of\\
\-\hspace{0.5cm} parameters within a geographic area of intrest\\
\-\hspace{0.5cm} https://docs.airnowapi.org/\\
\\
\end{tabular}
\end{flushleft}

}

%------------------------------------------------
% Input and Output Data
%------------------------------------------------

\headerbox{Input and Output Data}{name=input and output data,column=2,row=0}{
\begin{flushleft}
\begin{tabular}{@{}ll@{}}

$>$$>$$>$ act\_obj = act.io.create\_obj\_from\_arm\_dod(\\
\-\hspace{1.5cm} proc, set\_dims[,...])\\
\-\hspace{0.2cm} $\bullet$ Queries the ARM DOD api and builds an\\
\-\hspace{0.5cm} object based on the ARM DOD and the\\
\-\hspace{0.5cm} dimension sizes that are passed in.\\
\\
$>$$>$$>$ flag = act.io.check\_arm\_standards(act\_obj)\\
\-\hspace{0.2cm} $\bullet$ Checks to see if an xarray dataset conforms\\
\-\hspace{0.5cm} to ARM standards.\\
\\
$>$$>$$>$ act\_obj = act.io.read\_netcdf(filenames[,...])\\
\-\hspace{0.2cm} $\bullet$ Returns xarray.Dataset with stored data and\\
\-\hspace{0.5cm} metadata from a user-defined query of ARM-\\
\-\hspace{0.5cm} standard netCDF files from a single datastream.\\
\\
$>$$>$$>$ act.io.read\_csv(filename[,...])\\
\-\hspace{0.2cm} $\bullet$ Returns an xarray.Dataset with stored data and\\
\-\hspace{0.5cm} metadata from user-defined query of CSV files.\\
\\
$>$$>$$>$ act.io.read\_sigma\_mplv5(filename[,...])\\
\-\hspace{0.2cm} $\bullet$ Returns xarray.Dataset with stored data and\\
\-\hspace{0.5cm} metadata from a user-defined SIGMA MPL V5\\
\-\hspace{0.5cm} files.\\
\\
$>$$>$$>$ act.io.read\_gml(filename[,...])\\
\-\hspace{0.2cm} $\bullet$ Function to call or guess what reading NOAA\\
\-\hspace{0.5cm} GML daga routine to use.\\
\\
$>$$>$$>$ act.io.read\_gml\_co2(filename[,...])\\
\-\hspace{0.2cm} $\bullet$ Function to read carbon dioxide data from NOAA\\
\-\hspace{0.5cm} GML.\\
\\
$>$$>$$>$ act.io.read\_gml\_halo(filename[,...])\\
\-\hspace{0.2cm} $\bullet$ Function to read Halocarbon data from NOAA\\
\-\hspace{0.5cm} GML.\\
\\
$>$$>$$>$ act.io.read\_gml\_met(filename[,...])\\
\-\hspace{0.2cm} $\bullet$ Function to read meteorological data from\\
\-\hspace{0.5cm} NOAA GML.\\
\\
$>$$>$$>$ act.io.read\_gml\_ozone(filename[,...])\\
\-\hspace{0.2cm} $\bullet$ Function to read ozone data from NOAA GML.\\
\\
$>$$>$$>$ act.io.read\_gml\_radiation(filename)\\
\-\hspace{0.2cm} $\bullet$ Function to read radiation data from NOAA GML.\\
\\
$>$$>$$>$ act.io.read\_hk\_file(filename)\\
\-\hspace{0.2cm} $\bullet$ This procedure will read in an SP2 housekeeping\\
\-\hspace{0.5cm} file\\
\\
$>$$>$$>$ act.io.read\_psl\_parsivel(filename)\\
\-\hspace{0.2cm} $\bullet$ Returns xarray.Dataset with stored data and\\
\-\hspace{0.5cm} metadata from a defined NOAA PSL parsivel.\\
\\
$>$$>$$>$ act.io.read\_psl\_wind\_profiler(filename[,...])\\
\-\hspace{0.2cm} $\bullet$ Returns xarray.Dataset with stored data and\\
\-\hspace{0.5cm} metadata from a user-defined NOAA PSL\\
\-\hspace{0.5cm} wind profiler file.\\
\\
$>$$>$$>$ act.io.read\_psl\_wind\_profiler\_temperature(\\
\-\hspace{1.5cm} filename)\\
\-\hspace{0.2cm} $\bullet$ Returns xarray.Dataset with stored data and\\
\-\hspace{0.5cm} metadata from a user-defined NOAA PSL\\
\-\hspace{0.5cm} wind profiler temperature file.\\
\\
$>$$>$$>$ act.io.read\_sp2(filename[,...])\\
\-\hspace{0.2cm} $\bullet$ Loads a binary SP2 raw data file and returns all\\
\-\hspace{0.5cm} of the wave forms into an xarray Dataset.\\
\\
$>$$>$$>$ act.io.read\_sp2\_dat(filename[,...])\\
\-\hspace{0.2cm} $\bullet$ This reads the .dat files that generate the\\
\-\hspace{0.5cm} intermediate parameters used by the Igor\\
\-\hspace{0.5cm} processing. Wildcards are supported.\\
\\
\end{tabular}
\end{flushleft}

}

\end{poster}
\newpage

%%%%%%%%%%%%%%%%%%%%%%%%%%%%%%%%%%%%%%%%%%%%%%%%%%%%%%%%%%
%%%%%%%%%%%%%%%%%%    SECOND PAGE    %%%%%%%%%%%%%%%%%%%%%
%%%%%%%%%%%%%%%%%%%%%%%%%%%%%%%%%%%%%%%%%%%%%%%%%%%%%%%%%%

\begin{poster}
{
headerborder=closed, colspacing=0.8em, bgColorOne=white, bgColorTwo=white, borderColor=lightblue, headerColorOne=black, headerColorTwo=lightblue,
headerFontColor=white, boxColorOne=white, textborder=roundedleft, eyecatcher=true, headerheight=0.06\textheight, headershape=roundedright, headerfont=\Large\bf\textsc, linewidth=2pt
}
%----------------------------------------------------------------
%   Title
%----------------------------------------------------------------
{\bf\textsc{ACT Cheat Sheet}\vspace{0.5em}} % Poster title
{\textsc{\ A C T \ \ \ \ \ C h e a t \ \ \ \ \ S h e e t\ \hspace{12pt}}}
{\textsc{Learn More About ACT at https://arm-doe.github.io/ACT/ \hspace{12pt}}}

%------------------------------------------------
% Plotting
%------------------------------------------------

\headerbox{Plotting}{name=plotting,column=0,span=1,row=0}{
\begin{flushleft}

\begin{tabular}{@{}ll@{}}
\multicolumn{2}{l}{\cellcolor[HTML]{DDFFFF}\bf Display} \\
\\
Class that contains the common attributes and\\
routine between the differing Display classes.\\
\\
$>$$>$$>$ display = act.plotting.Display(\\
\-\hspace{1.2cm} obj, subplot\_shape=(1, ), ds\_name=None,\\
\-\hspace{1.2cm} subplot\_kw=None, **kwargs)\\
\\
$>$$>$$>$ display.add\_colorbar(\\
\-\hspace{1.2cm} mappable, title=None, subplot\_index=(0, ))\\
\-\hspace{0.2cm} $\bullet$ Adds a colorbar.\\
$>$$>$$>$ display.add\_subplots(\\
\-\hspace{1.2cm} subplot\_shape=(1, ), subplot\_kw=None,\\
\-\hspace{1.2cm} **kwargs)\\
\-\hspace{0.5cm} $\bullet$ Adds subplot to display object.\\
$>$$>$$>$ display.assign\_to\_figure\_axis(fig, ax)\\
\-\hspace{0.2cm} $\bullet$ This assigns the Display to a specific figure\\
\-\hspace{0.5cm} and axis.\\
$>$$>$$>$ display.put\_display\_in\_subplot(\\
\-\hspace{1.2cm} display, subplot\_index))\\
\-\hspace{0.2cm} $\bullet$ This will place a Display object into a specific\\
\-\hspace{0.5cm} subplot.
\end{tabular}

\begin{tabular}{@{}ll@{}}
\\
\multicolumn{2}{l}{\cellcolor[HTML]{DDFFFF}\bf TimeSeriesDisplay} \\
\\
This subclass contains routines that are specific to\\
plotting time series plots from data.\\
\\
$>$$>$$>$ display = act.plotting.TimeSeriesDisplay(\\
\-\hspace{1.2cm} obj, subplot\_shape=(1, ), ds\_name=None,\\
\-\hspace{1.2cm} **kwargs)\\
\\
$>$$>$$>$ display.plot(field[, ...])\\
\-\hspace{0.2cm} $\bullet$ Makes a timeseries plot.\\
$>$$>$$>$ display.plot\_barbs\_from\_spd\_dir(dir\_field[, ...]\\
\-\hspace{0.2cm} $\bullet$ This procedure will make a wind barb plot\\
\-\hspace{0.5cm} timeseries.\\
$>$$>$$>$ display.plot\_barbs\_from\_u\_v(u\_field, v\_field\\
\-\hspace{1.2cm}  [, ...])\\
\-\hspace{0.2cm} $\bullet$ This function will plot a wind barb timeseries\\
\-\hspace{0.5cm} from u and v wind data. If pres\_field is given, a\\
\-\hspace{0.5cm} height a time-height series will be plotted\\
\-\hspace{0.5cm} from 1-D wind data.\\
$>$$>$$>$ dis.plot\_time\_height\_xsection\_from\_1d\_data(\\
\-\hspace{1.2cm} data\_field, pres\_field[, ...])\\
\-\hspace{0.2cm} $\bullet$ This will plot a time-height cross section\\
\-\hspace{0.5cm} from 1D datasets using nearest neighbor\\
\-\hspace{0.5cm} interpolation on a regular time by height grid.\\
$>$$>$$>$ display.time\_height\_scatter(\\
\-\hspace{1.2cm} data\_field, dsname[, ...])\\
\-\hspace{0.2cm} $\bullet$ Create a time series plot of altitude and data\\
\-\hspace{0.5cm} variable with color also indicating value with a\\
\-\hspace{0.5cm} color bar.\\
\end{tabular}

\begin{tabular}{@{}ll@{}}
\\
\multicolumn{2}{l}{\cellcolor[HTML]{DDFFFF}\bf SkewTDisplay} \\
\\
A class for making Skew-T plots.\\
\\
$>$$>$$>$ display = act.plotting.SkewTDisplay(\\
\-\hspace{1.2cm} obj, subplot\_shape=(1, ), ds\_name=None,\\
\-\hspace{1.2cm} **kwargs)\\
\\
$>$$>$$>$ display.plot\_from\_spd\_and\_dir(\\
\-\hspace{1.2cm} spd\_field, dir\_field, p\_field, t\_field,\\
\-\hspace{1.2cm} td\_field[, ...])\\
\-\hspace{0.2cm} $\bullet$ This plot will make a sounding plot from wind\\
\-\hspace{0.5cm} data that is given in speed and direction.\\
\\
$>$$>$$>$ display.plot\_from\_u\_and\_v(\\
\-\hspace{1.2cm} u\_field, v\_field, p\_field, t\_field,\\
\-\hspace{1.2cm} td\_field[, ...])\\
\-\hspace{0.2cm} $\bullet$ This function will plot a Skew-T from a sounding\\
\-\hspace{0.5cm} dataset. The wind data must be given in u and v.\\
\\
\end{tabular}

\end{flushleft}
}
%------------------------------------------------
% Plotting Continued
%------------------------------------------------

\headerbox{Plotting Continued}{name=plotting,column=1,span=1,row=0}{
\begin{flushleft}
\begin{tabular}{@{}ll@{}}
\multicolumn{2}{l}{\cellcolor[HTML]{DDFFFF}\bf WindRoseDisplay} \\
\\
A class for handing wind rose plots..\\
\\
$>$$>$$>$ display = act.plotting.WindRoseDisplay(\\
\-\hspace{1.2cm} obj, subplot\_shape=(1, ), ds\_name=None,\\
\-\hspace{1.2cm} **kwargs)\\
\\
$>$$>$$>$ display.plot(dir\_field, spd\_field[, ...])\\
\-\hspace{0.2cm} $\bullet$ Makes the wind rose plot from the given dataset.\\
\\
$>$$>$$>$ display.plot\_data(dir\_field, spd\_field, data\_field)\\
\-\hspace{0.2cm} $\bullet$ Makes a data rose plot in line or boxplot\\
\-\hspace{0.5cm} form from the given data.\\
\end{tabular}

\begin{tabular}{@{}ll@{}}
\\
\multicolumn{2}{l}{\cellcolor[HTML]{DDFFFF}\bf XSectionDisplay} \\
\\
Plots cross sections of multidimensional datasets.\\
\\
$>$$>$$>$ display = act.plotting.XSectionDisplay(\\
\-\hspace{1.2cm} obj, subplot\_shape=(1, ), ds\_name=None,\\
\-\hspace{1.2cm} **kwargs)\\
\\
$>$$>$$>$ display.plot\_xsection(dsname, varname[, ...])\\
\-\hspace{0.2cm} $\bullet$ This function plots a cross section whose x and\\
\-\hspace{0.5cm} y coordinates are specified by the variable names.\\
\\
$>$$>$$>$ display.plot\_xsection\_map(\\
\-\hspace{1.2cm} dsname, varname[, ...])\\
\-\hspace{0.2cm} $\bullet$ Plots a cross section of 2D data on a geographical\\
\-\hspace{0.5cm} map.\\
\end{tabular}

\begin{tabular}{@{}ll@{}}
\\
\multicolumn{2}{l}{\cellcolor[HTML]{DDFFFF}\bf GeographicPlotDisplay} \\
\\
A class for making geographic tracer plot of aircraft,\\
ship or other moving platform plot.\\
\\
$>$$>$$>$ display = act.plotting.GeographicPlotDisplay(\\
\-\hspace{1.2cm} obj, ds\_name=None, **kwargs)\\
\\
$>$$>$$>$ display.geoplot(data\_field[, ...])\\
\-\hspace{0.2cm} $\bullet$ Creates a latitude and longitude plot of a time\\
\-\hspace{0.5cm} series data set with data values indicated by color.\\
\end{tabular}

\begin{tabular}{@{}ll@{}}
\\
\multicolumn{2}{l}{\cellcolor[HTML]{DDFFFF}\bf DistributionDisplay} \\
\\
Class used to make histogram plots.\\
\\
$>$$>$$>$ display = act.plotting.DistributionDisplay(\\
\-\hspace{1.2cm} obj, subplot\_shape=(1, ), ds\_name=None,\\
\-\hspace{1.2cm} **kwargs)\\
\\
$>$$>$$>$ display.plot\_heatmap(x\_field, y\_field[, ...])\\
\-\hspace{0.2cm} $\bullet$ This procedure will plot a heatmap of a histogram\\
\-\hspace{0.5cm} from 2 variables.\\
\\
$>$$>$$>$ display.plot\_stacked\_bar\_graph(field[, ...])\\
\-\hspace{0.2cm} $\bullet$ This procedure will plot a stacked bar graph of a\\
\-\hspace{0.5cm} histogram.\\
\\
$>$$>$$>$ display.plot\_stairstep\_graph(field[, ...])\\
\-\hspace{0.2cm} $\bullet$ This procedure will plot a stairstep plot of a\\
\-\hspace{0.5cm} histogram.\\
\end{tabular}

\end{flushleft}
}

%------------------------------------------------
% QC
%------------------------------------------------

\headerbox{QC}{name=qc,column=1,span=1,row=.85}{
\begin{flushleft}
\begin{tabular}{@{}ll@{}}
$>$$>$$>$ act.qc.add\_dqr\_to\_qc(obj[,...])\\
\-\hspace{0.2cm} $\bullet$ Function to query the ARM DQR web service\\
\-\hspace{0.5cm} control test for reports and add as a new quality\\
\-\hspace{0.5cm} to ancillary quality control variable.\\
\\
$>$$>$$>$ act.qc.apply\_supplemental\_qc(obj[,...])\\
\-\hspace{0.2cm} $\bullet$ Apply flagging from supplemental QC file\\
\-\hspace{0.5cm} by adding new QC tests.\\
\\
\end{tabular}
\end{flushleft}
}

%------------------------------------------------
% QC Continued
%------------------------------------------------

\headerbox{QC Continued}{name=qc,column=2,span=1,row=0}{
\begin{flushleft}
\begin{tabular}{@{}ll@{}}
Classes listed in blue have functions that can be\\
found in ACT's documentation:\\
https://arm-doe.github.io/ACT/API/index.html\\
\\
\multicolumn{2}{l}{\cellcolor[HTML]{DDFFFF}\bf CleanDataset} \\
\\
Class for cleaning up QC variables to standard\\
cf-compliance.\\
\\
$>$$>$$>$ act.qc.CleanDataset(obj)\\
\\
\multicolumn{2}{l}{\cellcolor[HTML]{DDFFFF}\bf QCFilter} \\
\\
Class for building quality control variables containing\\
arrays for filtering data based on a set of test condition\\
typically based on the values in the data fields.\\
\\
$>$$>$$>$ act.qc.QCFilter(obj)\\
\\
\multicolumn{2}{l}{\cellcolor[HTML]{DDFFFF}\bf QCTests} \\
\\
Method to perform a time series comparison test\\
between two Xarray Datasets to detect a shift in\\
time based on two similar variables.\\
\\
$>$$>$$>$ act.qc.QCTests(obj)\\
\\
\end{tabular}
\end{flushleft}
}

%------------------------------------------------
% Retrievals
%------------------------------------------------

\headerbox{Retrievals}{name=retrievals,column=2,span=1,row=.386}{
\begin{flushleft}
\begin{tabular}{@{}ll@{}}
$>$$>$$>$ act.retrievals.calculate\_stability\_indicies(\\
\-\hspace{1.2cm} ds[,...])\\
\-\hspace{0.2cm} $\bullet$ Calculates stability indices and adds it\\
\-\hspace{0.5cm} to the data set.
\\
$>$$>$$>$ act.retrievals.aeri2irt(\\
\-\hspace{1.2cm} aeri\_ds[,...])\\
\-\hspace{0.2cm} $\bullet$ This function will integrate over the correct\\
\-\hspace{0.5cm} wavenumber values to produce the effective IRT\\
\-\hspace{0.5cm} temperature.\\
\\
$>$$>$$>$ act.retrievals.calculate\_pbl\_liu\_liang(\\
\-\hspace{1.2cm} ds[,...])\\
\-\hspace{0.2cm} $\bullet$ Function for calculating the PBL height from a\\
\-\hspace{0.5cm} radiosonde profile using the Liu-Liang 2010\\
\-\hspace{0.5cm} technique.\\
\\
$>$$>$$>$ act.retrievals.calc\_sp2\_diams\_masses(\\
\-\hspace{1.2cm} ds[,...])\\
\-\hspace{0.2cm} $\bullet$ Calculates the scattering and incandescence\\
\-\hspace{0.5cm} diameters/BC masses for each particle.\\
\\
$>$$>$$>$ act.retrievals.calculate\_precipitable\_water(\\
\-\hspace{1.2cm} ds[,...])\\
\-\hspace{0.2cm} $\bullet$ Function to calculate precipitable water\\
\-\hspace{0.5cm} vapor from ARM sondewnpn b1 data.\\
\end{tabular}
\end{flushleft}

}

%------------------------------------------------
% Utilities
%------------------------------------------------

\headerbox{Utilities}{name=utilities,column=2,span=1,row=0.746}{
\begin{flushleft}
\begin{tabular}{@{}ll@{}}
$>$$>$$>$ ds = act.utils.create\_pyart\_obj(obj)\\
\-\hspace{0.2cm} $\bullet$ Produces a Py-ART object from an ACT object.\\
\\
$>$$>$$>$ ds = act.utils.calculate\_dqr\_times(obj[,...])\\
\-\hspace{0.2cm} $\bullet$ Function to retrieve start and end times of\\
\-\hspace{0.5cm} missing or bad data.\\
\\
$>$$>$$>$ ds = act.utils.height\_adjusted\_pressure(obj[,...])\\
\-\hspace{0.2cm} $\bullet$ Converts pressure for change in height.\\
\\
$>$$>$$>$ ds = act.utils.convert\_units(data, in\_units\\
\-\hspace{1.5cm} out\_units)\\
\-\hspace{0.2cm} $\bullet$ Converts units of a data array.\\
\\
$>$$>$$>$ ds = act.utils.accumulate\_precip(obj, variable)\\
\-\hspace{0.2cm} $\bullet$ Accumulate rain rates from an act object and\\
\-\hspace{0.5cm} insert variable back into act object.
\end{tabular}

\end{flushleft}

}

\end{poster}
\end{document}
